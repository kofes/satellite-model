\chapter{Вывод условий управления}
\section{Теневая зона орбиты}
Рассмотрим график нахождения спутника в плоскости $xOy$ (рис. \ref{fig:EarthShadow}),
где в качестве центра системы координат возьмем центр Солнца; ось $OX$ направлена
вдоль вектора, направленного из центра Солнца к центру Земли ($\vec{r}_{SE}$);
ось $OY$ лежит в плоскости, образованной векторами $\vec{r}_{SE}$ и $\vec{r}$ и направлена
вдоль вектора $(\vec{r}\times\vec{r}_{SE})\times\vec{r}$.
\begin{figure}[!h]%{r}{0pt}
\centering
\begin{tikzpicture}[thick,scale=0.6, every node/.style={transform shape}]
  \coordinate (Sun) at (0,0);
  \coordinate (Earth) at (16,0);
  \coordinate (Y) at (0,4);
  \coordinate (X) at (24,0);
  \coordinate (YDot) at (16,4);

  \draw[thick, ->] (-4, 0) -- (X) node[below] {$x$};
  \draw[thick, ->] (0, -4) -- (Y) node[left] {$y$};

  \draw[yellow, thick] (Sun) circle (3);
  \draw[->] (Sun) -- node[left] {$R_{Sun}$} (-2.1213203435596424, -2.1213203435596424);

  \draw[thick, ->] (Sun) -- node[below] {$\vec{r}_{SE}$} (Earth);

  \draw[thick, ->] (Earth) -- (16, 4) node[left] {$y'$};
  \draw[thick] (Earth) -- (16, -4);

  \draw[blue, thick] (Earth) circle (1);
  \draw[->] (Earth) -- node[right] {$R_{Earth}$} (15.292893218813452, -0.7071067811865475);

  \draw[thick, dashed, red] (Earth) circle (2);

  % спутник
  \draw[thick, ->] (Earth) -- (16.5176380902050415, 1.9318516525781366)  node[right] {$\vec{r}$};
  % угол спутника
  \draw (17.3, 0) arc (0:75:1.3) node[right] {$\phi$};

  % полутень
  \draw[thick, dashed] (0, 3) -- (24, -3);
  \draw[thick, dashed] (0, -3) -- (24, 3);
  \draw (0.7, -2.8) arc (20:90:0.7) node[right] {$\omega_{\text{пт}}$};
  % угол полутени
  \coordinate (Shadow2) at (17.461435894335995, 1.3653589735839988);
  \draw (16.7, 0) arc (0:45:0.7) node[right] {$\phi_{\text{пт}}$};
  \draw (16, 0) -- (Shadow2);

  % тень
  \draw[thick, dashed] (0, 3) -- (24, 0);
  \draw[thick, dashed] (0, -3) -- (24, 0);
  \draw (0.7, 2.9) arc (-15:-90:0.7) node[right] {$\omega_{\text{т}}$};
  % угол тени
  \draw (16.3, 0) arc (0:22.5:0.3) node[right] {$\phi_{\text{т}}$};
  \coordinate (Shadow) at (17.846153846153847, 0.7692307692307692);
  \draw (16, 0) -- (Shadow);
\end{tikzpicture}
\caption{Образование теневых зон}
\label{fig:EarthShadow}
\end{figure}\par
Здесь $R_{Sun}$ -- радиус Солнца; $R_{Earth}$ -- радиус Земли;
$\omega_{\text{т}}$; $\omega_{\text{пт}}$  -- вспомогательные углы, образующие
коническую поверхность областей тени и полутени; $\phi$ -- угол между вектором
$\vec{r}$ и $\vec{r}_{SE}$; $\phi_{\text{т}}$, $\phi_{\text{пт}}$ -- углы,
характеризующие линии раздела областей тени и полутени.\par
Угол $\phi$ находится через скалярное произведение векторов $\vec{r}$ и $\vec{r}_{SE}$
\begin{equation}
\vec{r} \cdot \vec{r}_{SE} = |\vec{r}||\vec{r}_{SE}|\cos\phi
\end{equation}
Тем самым, получим угол $\phi$
\begin{equation}
\phi = \arccos\frac{\vec{r} \cdot \vec{r}_{SE}}{|\vec{r}||\vec{r}_{SE}|}
\end{equation}\par
Углы $\omega_{\text{т}}$; $\omega_{\text{пт}}$ выводятся из следующих выражений:
\begin{equation}
\tg\omega_{\text{т}} = \frac{R_{Sun} - R_{Earth}}{r_{SE}},\,\,
\tg\omega_{\text{пт}} = \frac{R_{Sun} + R_{Earth}}{r_{SE}}.
\end{equation}
\par
Найдем углы $\phi_{\text{т}}$, $\phi_{\text{пт}}$, решив системы уравнений,
описывающих линии раздела областей тени, полутени и области без тени.\par
\begin{equation}
(1): \begin{cases}
  y = R_{Sun} - \tg\omega_{\text{т}} \cdot x, \\
  y^2 + (x - r_{SE})^2 = r^2, \\
  y = \tg\phi_{\text{т}} \cdot (x - r_{SE}),
\end{cases}
(2): \begin{cases}
  y = - R_{Sun} + \tg\omega_{\text{пт}} \cdot x, \\
  y^2 + (x - r_{SE})^2 = r^2, \\
  y = \tg\phi_{\text{пт}} \cdot (x - r_{SE}),
\end{cases}
\end{equation}\par
Решим первую систему, первоначально найдя крайнюю правую точку пересечения линии раздела областей
и орбиты спутника.\par
\begin{equation}
\begin{aligned}
   &\begin{cases}
    y = R_{Sun} - \tg\omega_{\text{т}} \cdot x, \\
    (R_{Sun} - \tg\omega_{\text{т}} \cdot x)^2 + (x - r_{SE})^2 = r^2, \\
    y = \tg\phi_{\text{т}} \cdot (x - r_{SE}),
  \end{cases} \\
  &\begin{cases}
    y = R_{Sun} - \tg\omega_{\text{т}} \cdot x, \\
    R_{Sun}^2 - 2 R_{Sun} \tg\omega_{\text{т}} \cdot x + \tg^2\omega_{\text{т}} \cdot x^2
    + x^2 - 2 r_{SE} x + r^2_{SE} = r^2, \\
    y = \tg\phi_{\text{т}} \cdot (x - r_{SE}),
  \end{cases} \\
  &\begin{cases}
    y = R_{Sun} - \tg\omega_{\text{т}} \cdot x, \\
    (\tg^2\omega_{\text{т}} + 1) \cdot x^2
    - 2 (r_{SE} + R_{Sun} \tg\omega_{\text{т}}) \cdot x
    + (R_{Sun}^2 + r^2_{SE} - r^2) = 0, \\
    y = \tg\phi_{\text{т}} \cdot (x - r_{SE}),
  \end{cases} \\
  & D/4 = (r_{SE} + R_{Sun} \tg\omega_{\text{т}})^2 -
  (\tg^2\omega_{\text{т}} + 1) (R_{Sun}^2 + r^2_{SE} - r^2), \\
  & D/4 = (\tg^2\omega_{\text{т}} + 1)r^2 - (R_{Sun}^2 - 2R_{Sun}r_{SE}\tg\omega_{\text{т}} + r^2_{SE}\tg^2\omega_{\text{т}}), \\
  & D/4 = (\tg^2\omega_{\text{т}} + 1)r^2 - (R_{Sun} - r_{SE}\tg\omega_{\text{т}})^2, \\
  & x = \frac{
    r_{SE} + R_{Sun} \tg\omega_{\text{т}} \pm \sqrt{(\tg^2\omega_{\text{т}} + 1)r^2 - (R_{Sun} - r_{SE}\tg\omega_{\text{т}})^2}
  }{\tg^2\omega_{\text{т}} + 1}, \\
  & \begin{cases}
    y_{\text{т}} = R_{Sun} - \tg\omega_{\text{т}} \cdot x_{\text{т}}, \\
    x_{\text{т}} = \frac{
      r_{SE} + R_{Sun} \tg\omega_{\text{т}} + \sqrt{(\tg^2\omega_{\text{т}} + 1)r^2 - (R_{Sun} - r_{SE}\tg\omega_{\text{т}})^2}
    }{\tg^2\omega_{\text{т}} + 1},
  \end{cases}
\end{aligned}
\end{equation}\par
Получив точку пересечения, выведем угол $\phi_{\text{т}}$:
\begin{equation}
\begin{aligned}
  & \begin{cases}
    \phi_{\text{т}} = \arctg\frac{y_{\text{т}}}{x_{\text{т}} - r_{SE}}, \\
    y_{\text{т}} = R_{Sun} - \tg\omega_{\text{т}} \cdot x_{\text{т}}, \\
    x_{\text{т}} = \frac{
      r_{SE} + R_{Sun} \tg\omega_{\text{т}} + \sqrt{(\tg^2\omega_{\text{т}} + 1)r^2 - (R_{Sun} - r_{SE}\tg\omega_{\text{т}})^2}
    }{\tg^2\omega_{\text{т}} + 1},
  \end{cases}
\end{aligned}
\end{equation}\par
Аналогично выводится угол полутени $\phi_{\text{пт}}$. Т.к. при подстановке уравнения
линии раздела в уравнение окружности теряется знак и получается аналогичное уравнение,
отличающееся от вышеописанного тем, что используется тангенс угла наклона $\omega_{\text{пт}}$,
вместо $\omega_{\text{т}}$.\par
Опустив выкладку, напишем формулу вычисления угла полутени:
\begin{equation}
\begin{aligned}
  & \begin{cases}
    \phi_{\text{пт}} = \arctg\frac{y_{\text{пт}}}{x_{\text{пт}} - r_{SE}}, \\
    y_{\text{пт}} = - R_{Sun} + \tg\omega_{\text{пт}} \cdot x_{\text{пт}}, \\
    x_{\text{пт}} = \frac{
      r_{SE} + R_{Sun} \tg\omega_{\text{пт}} + \sqrt{(\tg^2\omega_{\text{пт}} + 1)r^2 - (R_{Sun} - r_{SE}\tg\omega_{\text{пт}})^2}
    }{\tg^2\omega_{\text{пт}} + 1},
  \end{cases}
\end{aligned}
\end{equation}\par
В первом приближении, можно использовать в качестве функции тени, следующую модель:
\begin{equation}
f(r, \phi) = \begin{cases}
  0, & \phi < \phi_{\text{т}}, \\
  0.5, & \phi_{\text{т}} < \phi < \phi_{\text{пт}}, \\
  1, & \phi > \phi_{\text{пт}},
\end{cases}
\end{equation}
однако, в таком случае переход из одного состояние в другое не будет плавным, поэтому
вводятся различные модели плавного перехода из одной области в другую. В данной работе,
используется <<сигмоид>> следующего вида:
\begin{equation}
    f(\phi, \phi_{\text{пт}}, \phi_{\text{т}}) = \frac{1}{1 + e^{-2\frac{\phi - \phi_{\text{т}}}{\phi_{\text{пт}} - \phi_{\text{т}}}}}.
\end{equation}
Тогда ускорение, приобретаемое телом под действием силы солнечного давления, примет
следующий вид:
\begin{equation}
    \vec{f}_{\text{SRP}} = \frac{\vec{F}_{\text{SRP}}}{m} \cdot f(\phi, \phi_{\text{пт}}, \phi_{\text{т}}).
\end{equation}
\section{Определение области допустимых значений управляющих углов}
\noindent\indent Рассмотрим уравнения (\ref{eq:EulerDynamic})-(\ref{eq:EulerKinematic}).
Переобозначим матрицу $W$ через кососимметрическую матрицу замены векторного произведения.
\begin{equation}
    W = -[\vec{\omega}_{\text{отн}}]_{\times},
\end{equation}
где
\begin{equation}
    \begin{aligned}
        & \vec{a}\times\vec{b} = [\vec{a}]_{\times}\vec{b}, \\
        & [\vec{a}]_{\times} = \begin{bmatrix}
            0 & -a_3 & a_2 \\
            a_3 & 0 & -a_1 \\
            -a_2 & a_1 & 0
        \end{bmatrix},
    \end{aligned}
\end{equation}
в правоориентированной системе координат.\par
Подставим значения из кинематического уравнения в динамические уравнений Эйлера
и запишем вместо абсолютной угловой скорости его разложение. Получим выражение
следующего вида:
\begin{equation}
    \begin{aligned}
        & J\dot{\vec{\omega}}_{\text{отн}} - J[\vec{\omega}_{\text{отн}}]_{\times} A \vec{\omega}_0 + J A \dot{\vec{\omega}}_0 \\
        & + [\vec{\omega}_{\text{отн}}]_{\times} J \vec{\omega}_{\text{отн}} + [\vec{\omega}_{\text{отн}}]_{\times} J A \vec{\omega}_0 \\
        & + [A\vec{\omega}_0]_{\times} J \vec{\omega}_{\text{отн}} + [A\vec{\omega}_0]_{\times} J A \vec{\omega}_0 \\
        & = \vec{M}_{\text{вш}} + \vec{M}_{\text{упр}}.
    \end{aligned}
\end{equation}
Данное уравнение можно записать в более удобной матричной форме:
\begin{equation}
    \begin{bmatrix}
        [A\vec{\omega}_0]_{\times} J + [\vec{\omega}_{\text{отн}}]_{\times} J - J[\vec{\omega}_{\text{отн}}]_{\times} \\
        J
    \end{bmatrix}^T A \begin{bmatrix}
        \vec{\omega}_0 \\
        \dot{\vec{\omega}}_0
    \end{bmatrix} + \begin{bmatrix}
        [A\vec{\omega}_0]_{\times} + [\vec{\omega}_{\text{отн}}]_{\times} \\
        I
    \end{bmatrix}^T J \begin{bmatrix}
        \vec{\omega}_{\text{отн}} \\
        \dot{\vec{\omega}}_{\text{отн}}
    \end{bmatrix}
    = \vec{M}_{\text{вш}} + \vec{M}_{\text{упр}},
\end{equation}\par
    Отсюда возможно вывести зависимость относительного углового ускорения от
управляющего момента.\par
\begin{equation}
    \begin{aligned}
        \dot{\vec{\omega}}_{\text{отн}} =&
            - \begin{bmatrix}
                J^{-1}[A\vec{\omega}_0]_{\times} J + J^{-1}[\vec{\omega}_{\text{отн}}]_{\times} J - [\vec{\omega}_{\text{отн}}]_{\times} \\
                I
            \end{bmatrix}^T A \begin{bmatrix}
                \vec{\omega}_0 \\
                \dot{\vec{\omega}}_0
            \end{bmatrix} \\
        & -J^{-1} \left\{
            [A\vec{\omega}_0]_{\times} + [\vec{\omega}_{\text{отн}}]_{\times} J \vec{\omega}_{\text{отн}}
            - \vec{M}_{\text{вш}}
        \right\}
        + J^{-1}\vec{M}_{\text{упр}}
    \end{aligned}
\end{equation}
