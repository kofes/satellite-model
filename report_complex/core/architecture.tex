\chapter{Архитектура системы}
\subsection{Общая архитектура}
\noindent\indent Для разработки данных модулей требовалось, в первую очередь,
возможность его проверки и тестирования не только в качестве численных данных, но
и визуальной оценки правдоподобности их работы. Так как целевой продукт жестко привязан
к ОС Windows и имеет достаточно нагруженную архитектуру, было решено создать более
легковесную систему для визуализации данных, способную запускаться на ОС Windows и
дистрибутивов Linux с возможностью быстрого отделения этой системы от необходимых
к разработке модулей.
Ввиду чего, программное решение можно разделить на следующие компоненты:
\begin{itemize}
  \item Модуль расчета действующих сил и моментов
  \item Модуль подбора оптимального управления
  \item Визуализатор данных
\end{itemize}
\subsection{Модуль расчета действующих сил и моментов}
\noindent\indent
