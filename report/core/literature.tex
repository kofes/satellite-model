\chapter{Список литературы}
% \addcontentsline{toc}{chapter}{Список литературы}
\begin{enumerate}
    \item[1] Ф. А. Цандер Перелеты на другие планеты // Техника и жизнь. — 1924. — № 13 — С. 15—16.
    \item[2] Маринер-10 // Википедия. [2017—2017]. Дата обновления: 24.04.2017. URL: https://ru.wikipedia.org/?oldid=85032943 (дата обращения: 24.04.2017).
    \item[3] Dauna Coulter A Brief History of Solar Sails // NASA Science. URL: https://science.nasa.gov/science-news/science-at-nasa/2008/31jul\_solarsails
    (дата обращения: 17.06.2018)
    \item[4] Sigurd De Keyser Orbital Selected To Build And Launch NASA's New Millennium Space Technology 8 Satellite // The International Space Fellowship
    URL: https://www.spacefellowship.com/news/art1294/orbital-selected-to-build-and-launch-nasa-039-s-new-millennium-space-technology-8-satellite.html
    (дата обращения: 16.06.2018)
    \item Ф. А. Цандер Перелеты на другие планеты // Техника и жизнь. — 1924. — № 13 — С. 15—16.
    \item Белецкий В. В. Очерки о движении космических тел. 2-е изд. М.: Наука, 1977. 432 c.
    \item Аксенов Е. П. Теория движения искусственных спутников Земли. М.: Наука. 1977, 360 с.
    \item McInnes C. R. Solar Sailing: Technology, Dynamics and Mission Applications. Springer Science \&
Business Media, 2004. ISBN: 978-3-540-21062-7.
    \item Неровный Н. А., Зимин В. Н. Об определении силы светового давления на солнечный парус
с учетом зависимости оптических характеристик материала паруса от механических напряжений //
Вестник МГТУ им. Н. Э. Баумана. Сер. «Машиностроение». 2014. Т. 96, No 3. С. 61–78.
    \item Неровный Н. А. Главный вектор и главный момент светового давления на оптически выпуклую космическую конструкцию //
Вестник СПбГУ. Математика. Механика. Астрономия. Т. 4 (62). 2017. Вып. 1. C. 146-158.
    \item Чумаченко Е. Н., Малашкин А. В., Федоренко А. Н. Моделирование использования солнечного ветра
для орбитальных маневров космических аппаратов // Вестник ВГТУ. 2011. №11-2.
URL: https://cyberleninka.ru/article/n/modelirovanie-ispolzovaniya-solnechnogo-vetra-dlya-orbitalnyh-manevrov-kosmicheskih-apparatov (дата обращения: 21.04.2018).
    \item Неровный Н. А., Зимин В. Н., Об определении силы светового давления на солнечных парус
с учетом зависимости оптических характеристик материала паруса от механических напряжений //
Вестник МГТУ им. Н.Э. Баумана. Сер. “Машиностроение” 2014. No 3. C. 61-78.
    \item Трофимов С. П. Увод малых космических аппаратов с верхнего сегмента низких орбит
с помощью паруса для увеличения силы светового давления, Препринты ИПМ им. М. В. Келдыша, 2015, 032, 32 с.
    \item Распопова Н. В., Давыденко А. А. Задачи движения тел в космических системах.
Часть 1: Учеб. пособие / СПб.: ”СОЛО”, 2015. — 73 с.
    \item Мирер С. А. Механика космического полета. Орбитальное движение: учебное пособие /
М.: "Аскери Информэйшн", 2007. -- 272 с.
    \item Окишев Ю. А., Клинаев Ю. В. Математическое моделирование частной ограниченной
задачи трех тел с учётом второй зональной гармоники в геоцентрической экваториальной
системе координат // Вестник СГТУ. 2013. №1 (73). URL:
https://cyberleninka.ru/article/n/matematicheskoe-modelirovanie-chastnoy-ogranichennoy-zadachi-treh-tel-s-uchyotom-vtoroy-zonalnoy-garmoniki-v-geotsentricheskoy
(дата обращения: 12.05.2018).
    \item Powers R. B., Coverstone-Carrol V. L., Prussing J. E Solar sail optimal
orbit transfers to synchronous orbits // Advances in the Astronautical Sciences, 1999. Vol. 103. С. 523-538.
    \item Fuchang G, Lixing H. Implementing the Nelder-Mead simplex algorithm
with adaptive parameters. Springer Science \& Business Media, 2010. DOI 10.1007/s10589-010-9329-3.
    \item Ткачев С. С. Исследование управляемого углового движения аппаратов с
ротирующими элементами: дисс. ... кандидата физико-математических наук. Ин-т прикладной математики им. М.В. Келдыша РАН,
Москва, 2011.
    \item Л.Г.Лукьянов, Г.И. Ширмин. Лекции по небесной механике. Эверо. Алматы, 2009.
\end{enumerate}
