\chapter{Реализация и тестирования}
\noindent\indent В ходе разработки написано 6493 строки кода на языке C++,
98 строк кода на шейдерном языке GLSL. Проведены численные эксперименты.
\section{Тестирование методов оптимизации}
\noindent Для проверки правильности реализации методов, они были протестированы на
нескольких известных функциях: линейной, сферической и функции Розенброка.\par
Количество аргументов: 3.\par
Количество запусков: 500.\par
\begin{table}[ht]
\centering
\resizebox{\textwidth}{!}{\begin{tabular}{|l|c|c|c|}
    \hline
    Целевая функция-Алгоритм & Градиентный метод   & Генетический алгоритм & Алгоритм Нелдера-Мида \\ \hline
    Линейная функция         & 0                   & 0                     & 0                     \\ \hline
    Сферическая функция      & 0                   & 0                     & 0                     \\ \hline
    Функция Розенброка       & 500                 & 57                    & 0                     \\ \hline
\end{tabular}}
\caption{Результаты тестирования алгоритмов оптимизации}
\end{table}\par
% \chapter{Вычислительный эксперимент}
