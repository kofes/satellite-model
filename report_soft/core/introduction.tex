\usetikzlibrary{calc}
\usetikzlibrary{intersections}
\usetikzlibrary{angles}
\usetikzlibrary{quotes}

\chapter{Введение}
\noindent\indent Место прохождения практики: Центр проектной деятельности
Дальневосточного федерального универститета.\par
    Период прохождения практики: с 16 апреля 2018 г. по 8 июня 2018 г.\par
    Целью практики явлись моделирование процесса движения космического аппарата
в поле действия возмущающих сил и подборе алгоритма оптимального управления
ориентацией аппарата с целью увода спутника на более высокую орбиту.\par
    Были поставлены следующие задачи: программная реализация алгоритма нахождения
оптимального управления ориентацией аппарата и модуля определения возмущающих сил и
моментов действующих со стороны солнечного давления в каждый момент времени.\par
    Программа практики содержала создание гибкой тестирующей системы, наглядно
демонстрирущей работу системы моделирования процессов, связанных с положением и
вращением объекта работы, создание и тестирование модели действущей силы
солнечного давления в областях полного света, полной тени и в области полутени,
а также программная реализация алгоритмов подбора оптимальных углов ориентирования
рассматриваемого объекта.
