\section{Спецификация данных}
...
\section{Функциональные требования}
\noindent\indent Разрабатываемая система должна:
\begin{enumerate}
  \item Определять силы и моменты сил действующие на парус в каждый момент времени
  \item Определять углы оптимального управления
\end{enumerate}
\section{Проект}
\subsection{Средства реализации}
\noindent\indent При разработке использовался язык программирования C/C++,
как основной язык разработки системы моделирования <<Sputnix Satellite Modeller>>.
\subsection{Модули и алгоритмы}
\subsubsection{Прогнозирование оптимального <<положения>>}
\noindent\indent Предполагаем, что мы имеем возможность считать углы оптимального
управления в каждый момент времени для максимизации подынтегральной функции за довольно
малые промежутки времени. Тем самым, с течением времени, мы получим ряды, состоящие
из оптимальных углов и тех, что имеются на самом деле:
\begin{equation}
  \begin{aligned}
    &\{\theta_{t_i}\}_{i=\overline{1,N}} = \theta_{t_0}, \theta_{t_1}, ..., \theta_{t_N}, \\
    &\{\hat{\theta}_{t_i}\}_{i=\overline{1,N}} = \hat{\theta}_{t_0}, \hat{\theta}_{t_1}, ..., \hat{\theta}_{t_N}, \\
    &\{\phi_{t_i}\}_{i=\overline{1,N}} = \phi_{t_0}, \phi_{t_1}, ..., \phi_{t_N} \\
    &\{\hat{\phi}_{t_i}\}_{i=\overline{1,N}} = \hat{\phi}_{t_0}, \hat{\phi}_{t_1}, ..., \hat{\phi}_{t_N}, \\
  \end{aligned}
\end{equation}
здесь $\hat{\theta}_{t_i}$, $\hat{\phi}_{t_i}$ -- оптимальные углы управления аппаратом,
высчитанные для момента времени $t_i$; $\theta_{t_i}$, $\phi_{t_i}$ -- углы, которыми
ориентирована система в момент времени $t_i$.\par
  В таком случае, для наибольшей максимизации функционала, необходимо, чтобы
аппарат мгновенно менял свою ориентацию на наиболее оптимальную, что в условиях
существования внешних возмущающих моментов сил и собственного вращающего момента
спутника трудно достижимо.\par
  Однако, имея выборку уже вычисленных значений, возможно приближенно спрогнозировать
наиболее оптимальные угловые коэффициенты, которые должен будет принять аппарат
через некоторый промежуток времени $\Delta t$ и сделать управляющую функцию изменения
углов более плавной для большей эффективности системы автоматической ориентации КА.\par
  Пусть $\vec{W} = (\phi, \theta)^T$ -- вектор углов. Разложим значение данного вектора
в точке $t_{k+1}$ через ряд Тейлора в точке $t_{k}$:
\begin{equation} \label{eq:parialWEquatation}
  \vec{W}_{t_{k+1}} = \vec{W}_{t_{k}}
  + \frac{\partial}{\partial t}\vec{W}_{t_{k}} \cdot (t_{k+1} - t_{k})
  + \frac{\partial^2}{\partial t^2}\vec{W}_{t_{k}} \cdot (t_{k+1} - t_{k})^2
  + o(|t_{k+1} - t_{k}|^3)
\end{equation}\par
  Так как нам известно оптимальное значение углов, которое <<мгновенно>> необходимо
придать в каждой точке отсчета, значит, мы знаем <<мгновенную>> скорость, которую
будет пытаться придать система стабилизации КА:
\begin{equation}
  \frac{\partial}{\partial t}\vec{W}_{t_{k}} = \hat{\vec{W}}_{t_{k}} - \vec{W}_{t_{k}},
\end{equation}\par
  Также возможно получить приближенное значение дифференциала второго порядка путем
усреднения значений пересчетов между ближайшими узлами расчета:
\begin{equation}
  \frac{\partial^2}{\partial t^2}\vec{W}_{t_{k}} = \frac{1}{2}\left[
    \frac{\frac{\partial}{\partial t}\vec{W}_{t_{k+1}} - \frac{\partial}{\partial t}\vec{W}_{t_{k}}}{t_{k+1} - t_{k}}
    +
    \frac{\frac{\partial}{\partial t}\vec{W}_{t_{k}} - \frac{\partial}{\partial t}\vec{W}_{t_{k-1}}}{t_{k} - t_{k-1}}
  \right]
\end{equation}\par
  Тем самым мы получаем возможность предсказывать приближенное значение углов,
ориентирующих аппарат в заданный момент времени.\par
  Однако, есть ограничения на данную конструкцию. Как описывалось выше, управляющие
углы зависят не только от времени, но и от взаимного положения КА, Земли и Солнца,
тем самым разложение в ряд Тейлора примет следующий вид:
\begin{equation} \label{eq:FullWEquatation}
  \vec{W}_{t_{k+1,\vec{K}_{k+1}}} = \vec{W}_{t_{k},\vec{K}_{k}}
  + \left[
    \frac{\partial}{\partial t}\vec{W}_{t_{k},\vec{K}_{k}} \cdot (t_{k+1} - t_{k})
  + \frac{\partial}{\partial \vec{K}_k}\vec{W}_{t_{k},\vec{K}_{k}} \cdot (\vec{K}_{k+1} - \vec{K}_{k})
  \right]
  + o(||(t_{k+1} - t_{k}, \vec{K}_{k+1} - \vec{K}_{k})||^2),
\end{equation}\par
  В таком случае, для упрощения модели (\ref{eq:FullWEquatation}) до вида (\ref{eq:parialWEquatation})
необходимо ввести некоторые ограничения для более приближенного решения, к примеру:
\begin{equation} \label{eq:FullWApprox}
  \begin{aligned}
    &||(t_{k+1} - t_{k}, \vec{K}_{k+1} - \vec{K}_{k})||^2 - ||t_{k+1} - t_{k}||^2 < \delta,\\
    &||\frac{\partial}{\partial \vec{K}_k}\vec{W}_{t_{k},\vec{K}_{k}}|| < \epsilon,
  \end{aligned}
\end{equation}
где $\epsilon$ и $\delta$ -- некоторые специально подобранные константы.\par
  С целью уменьшения количества расчетов, возможна замена условий из (\ref{eq:FullWApprox})
на более жесткое:
\begin{equation}
  ||\vec{K}_{k+1} - \vec{K}_{k}|| < \zeta,
\end{equation}
где $\zeta$ -- специально подобранный коэффициент.
\subsubsection{Оптимизация на больших участков}
\noindent\indent Для нахождения оптимального управляющего углового функционала,
для некотором интервале времени, необходимо решить задачу оптимизации нелинейной
интегральной функции (\ref{eq:IntMaxFullEq}). Для этого, представим угловую функцию
в следующем виде:
\begin{equation}
  \begin{aligned}
    & \vec{W}(t) = (\phi(t), \theta(t))^T, \\
    & \vec{W}(t) = \vec{W}_0 + \vec{W}_1 t + \frac{1}{2}\vec{W}_2 t^2, \\
    & \vec{W}_0, \vec{W}_1, \vec{W}_2 \in [-\pi, \pi],
  \end{aligned}
\end{equation}
где $\vec{W}_0$, $\vec{W}_1$, $\vec{W}_2$ -- -- коэффициенты, которые необходимо найти.\par
  Возможно несколько подходов к решению данной задачи, в частности, метод наименьших
квадратов, генетический алгоритм, алгоритм Нелдера-Мида и градиентный метод.
